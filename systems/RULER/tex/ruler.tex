\documentclass{article}
\usepackage[utf8]{inputenc}
\usepackage[margin=0.6in]{geometry}
\usepackage[fontsize=12pt]{fontsize}
\usepackage{multicol}
\usepackage{hyperref}
\usepackage{graphicx} % Required for inserting images
\usepackage{CJKutf8} % Required for CJK
\usepackage[dvipsnames]{xcolor} % Required for \textcolor
\usepackage{pxfonts} % Required for solid diamond / heart symbol
\usepackage{enumitem}
\setlist[itemize,1]{
    topsep=-0.25em,
    itemsep=-0.25em, left=0.1em,
    label=$\blacktriangleright$,
    after=\vspace{0.5em}
}
\setlist[itemize,2]{
    topsep=-0.6em,
    itemsep=-0.25em, left=0.1em,
    label=$\vartriangleright$
}
\newcommand*{\ccc}{\textcolor{ForestGreen}{\clubsuit}}
\newcommand*{\ddd}{\textcolor{Orange}{\vardiamondsuit}}
\newcommand*{\hhh}{\textcolor{Mahogany}{\varheartsuit}}
\newcommand*{\sss}{\textcolor{Blue}{\spadesuit}}

\title{RULER Uses Level-Elastic Relays}
\author{O.Y. Chung $\hhh\sss\ccc\ddd$}
\date{June 2023}

\begin{document}

\begin{CJK*}{UTF8}{gbsn}
\end{CJK*}
\maketitle

\begin{multicols}{2}

\section{Abstract}
Exploiting relay patterns not tied to level, RULER supports full shape relays in both constructive or competitive auction with minimal rules to memorize.

\section{Shape Listing}
Near the end of relay, where the shape showing side has only a few possible shapes (typically 5 or less), the next relay will be listing exact shapes. For simplicity, the shape priority is defined as:

\begin{enumerate}
    \setlength\itemsep{-0.2em}
    \item more extreme shapes bids higher, where "more extreme" means:
    \begin{itemize}
        \setlength\itemsep{-0.2em}
        \item longest suit longer \\
            e.g. $6610 \rightarrow 7222$
        \item second longest suit longer \\
            e.g. $5440 \rightarrow 5521$
        \item third longest suit longer \\
            e.g. $5431 \rightarrow 5540$
    \end{itemize}
    \item with same shape, longing higher suits bids higher
\end{enumerate}

\noindent Shape answering is bounded by 3NT. When out of steps within 3NT, overload the uncoverred shapes on shapes within 3NT with nearest major length. For example, when knowing 6+$\ddd$, 0-1$\ccc$s:
\begin{itemize}
    \setlength\itemsep{-0.2em}
    \item 3$\hhh$ = 3361 or 3370
    \item 3$\sss$ = 2371 or 2281 or 1381 or 2380
    \item 3NT = 3271 or 3181 or 3280
\end{itemize}

\section{Elastic Relay Patterns}

\subsection{General Principle}
\begin{itemize}
    \setlength\itemsep{-0.2em}
    \item for GI+ relay, reserve \#1 for min \\
        for 2X open, reserve 3X (rebid) for min,
        that implies answers below 3X can be min or max
    \item 3$\ddd$ is the highest relay bid
    \item during relay and before exact shape shown: \\
        use 3$\hhh/\sss$ to probe 4$\hhh/\sss$
    \item do not answer higher than 3NT \\
        (reserve 4$\ccc$+ for auto-kc answer)
    \item when no more to relay, example scheme is:
    \begin{itemize}
        \setlength\itemsep{-0.2em}
        \item use 4$\ccc$ for any SI
        \item use lowest steps (one for each shown 4+card suit) \\
            for RKCB (do not use 3NT/4$\hhh$/4$\sss$)
    \end{itemize}
\end{itemize}

\subsection{444766}\label{sec:444766}
Bidding starts with showing one suit. If game-forcing power is promised, next step is to know the second suit.
Suit order is $\hhh \rightarrow \sss \rightarrow \ccc \rightarrow \ddd$. If there is no second suit, it auto-zooms to the 1-suiter cases. To summarize, the steps are:

\begin{enumerate}
    \setlength\itemsep{-0.2em}
    \item \textbf{\underline{4}}+ cards in in suit 1, cont' as 
    \nameref{sec:R564}
    \item \textbf{\underline{4}}+ cards in in suit 2, cont' as
    \nameref{sec:R564}
    \item \textbf{\underline{4}}+ cards in in suit 3, cont' as
    \nameref{sec:R564}
    \item \textbf{\underline{7}}+ cards in known suit, cont' as
    \nameref{sec:111}
    \item \textbf{\underline{6}} cards in known suit, 6-(331)
    \item \textbf{\underline{6}} cards in known suit, 6-(322)
    \item (and onwards) auto zoom balanced shapes \\
        (4-333/5-233/323/332)
\end{enumerate}

\pagebreak

\subsubsection{4446 when lack of space}\label{sec:4446}
In lack of space, the step for \textbf{\underline{7}} of 444766 must not be higher than 2NT. The set of windows will be changed to 4446, where 6 means 6 or more cards in the known suit, cont' as \nameref{sec:111}.

\noindent Since relay is bounded at 3NT, 3NT will become catch all in lack of space. For example, when relay bid is $3\ddd$, known suit is $\sss$:
\begin{itemize}
    \setlength\itemsep{-0.2em}
    \item 3$\hhh$ = 4+$\hhh$
    \item 3$\sss$ = 4+$\ccc$
    \item 3NT = 4+$\ddd$ or 6+$\sss$ or 4-(333) or 5-(332)
\end{itemize}

\subsection{R564}\label{sec:R564}
After any 4 of \nameref{sec:444766}. \\
After 2 suits are shown, 4 windows are used to break down the 2-suit cases. Inapplicable window should not occupy a step, for example, in RULER, the \textbf{\underline{R}}everser step is inapplicable.

\begin{enumerate}
    \setlength\itemsep{-0.2em}
    \item \textbf{\underline{R}}everser (higher suit is longer)
    \item 5-\textbf{\underline{5}} (inapplicable if bid higher suit first)
    \item \textbf{\underline{6}}-4
    \item \textbf{\underline{4}}-(4 or shorter) (inapplicable if 5+)
    \item (and onwards) auto zoom 5-4 shapes \\
        (5-4-22/13/31/04/40)
\end{enumerate}

\noindent The 3 windows 564 are only applicable when it is at 3$\ddd$ or below.

\subsection{111}\label{sec:111}
After 7 of \nameref{sec:444766} or 6 of \nameref{sec:4446}. \\
After 1-suiter is shown, 3 windows are used to show shortage. Suit order for shortage is $\sss \rightarrow \hhh \rightarrow \ddd \rightarrow \ccc$.

\begin{itemize}
    \setlength\itemsep{-0.2em}
    \item \#1/2/3 = 0-\textbf{\underline{1}} cards in suit 1/2/3
    \item \#4+ = auto zoom 6-(322) and 7222
\end{itemize}

\subsection{1NT To-Play}\label{sec:1nt-to-play}
Happens after 1$\ccc/\ddd/\hhh$-1NT-2$\ccc$. The scheme is to reduce 3 cards from a total of 16 max suit length (4444/4435/4255). Use of this scheme is an optional advancement. It can be replaced by saying opener's hand is more specific.

\columnbreak

\noindent Here are the response to the 2$\ccc$ relay:
\begin{itemize}
    \setlength\itemsep{-0.2em}
    \item 2$\ddd$ = min
    \item 2$\hhh$ = major suits reduce 0-1 cards
    \item 2$\sss$+ = major suits reduce 2-3 cards
\end{itemize}

\noindent For major suits reduce 0-1 cards:
\begin{itemize}
    \setlength\itemsep{-0.2em}
    \item 2NT = $\hhh$ longer than $\sss$ ($\sss$ -1) \\
        To 1$\ccc$-1NT (4444): 3433/3424/3442 \\
        To 1$\ddd$-1NT (4435): 3433/3424/3415 \\
        To 1$\hhh$-1NT (4255): 3244/3235/3253
    \item 3$\ccc$ = $\sss$ longer than $\hhh$ ($\hhh$ -1)
    \item 3$\ddd$+ = major suits reduces similar length \\
        (major suits reduces 0 cards)
\end{itemize}

\noindent For major suits reduce 2-3 cards:
\begin{itemize}
    \setlength\itemsep{-0.2em}
    \item 2$\sss$ = $\hhh$ longer than $\sss$ ($\sss$ -2/-3) \\
        To 1$\ccc$-1NT (4444): 2434/2443/1444 \\
        To 1$\ddd$-1NT (4435): 2434/2425/1435 \\
        To 1$\hhh$-1NT (4255): 2245/2254/1255
    \item 2NT = $\sss$ longer than $\hhh$ ($\hhh$ -2/-3)
    \item 3$\ccc$+ = major suits reduces similar length \\
        (each major suit reduces at least 1 card)
\end{itemize}

\section{Inelastic Relay Patterns}

\subsection{2$\hhh$ to 3NT Balanced}
\label{sec:2h-3nt-bal}

\begin{itemize}
    \setlength\itemsep{-0.2em}
    \item 2$\hhh$ = 4-5 $\hhh$s
    \item 2$\sss$ = 4-5 $\sss$s, not 4333
    \item 2NT = suit lengths $<=$ 3344
    \item 3$\ccc$ = 5 $\ccc$s
    \item 3$\ddd$ = 4333
    \item 3$\hhh$ to 3NT = 5 $\ddd$s
\end{itemize}

\noindent After 2$\hhh$ to 2$\sss$:
\begin{itemize}
    \setlength\itemsep{-0.2em}
    \item 2NT = 3433/4423/4432
    \item 3$\ccc$ = 5-(332)
    \item 3$\ddd$ to 3NT = 4-(234/324/243/342)
\end{itemize}

\noindent Advanced usage: when there is no 4441 cases, let 2$\ccc$ relay include a 5+$\hhh$ GI case. Adjust as:
\begin{itemize}
    \setlength\itemsep{-0.2em}
    \item move the 2$\hhh$ case by 2NT+ (auto accept GI)
    \item split the 2$\sss$+ case as 2$\hhh$=min; 2$\sss$=max
    \item move the 2NT+ case to 2$\ddd$, then partner 2$\hhh$ to show $\hhh$ GI case
\end{itemize}

\subsection{2$\hhh$ to 3NT Unbalanced}
\label{sec:2h-3nt-unbal}
Used only for hands with shortage. Not for hands like 5422, 6322 or 7222.

\begin{itemize}
    \setlength\itemsep{-0.2em}
    \item 2$\hhh$ = 1/2-suiter with $\hhh$
    \item 2$\sss$ = 1/2-suiter with $\sss$
    \item 2NT = 1/2-suiter with $\ccc$
    \item 3$\ccc$ = 1-suiter with $\ddd$
    \item 3$\ddd$+ = (4441)
\end{itemize}

\noindent After 2$\hhh$ to 3$\ccc$:
\begin{itemize}
    \setlength\itemsep{-0.2em}
    \item 2NT = 2-suiter with $\sss$, cont' as \nameref{sec:R564}
    \item 3$\ccc$ = 2-suiter with $\ccc$, cont' as \nameref{sec:R564}
    \item 3$\ddd$ = 2-suiter with $\ddd$ \\
        this is an exceptional case that we use 3$\hhh$ to relay for the shortage and use 3$\sss$ to probe 4M (if the other suit is $\hhh$ or $\sss$). One reason is we saved the step for 5-4-2-2 since that is inapplicable
    \item 3$\hhh$ = 1-suiter with suit 1 shortage
    \item 3$\sss$ = 1-suiter with suit 2 shortage
    \item 3NT = 1-suiter with suit 3 shortage
\end{itemize}

\section{Opening Scheme}
\begin{itemize}
    \setlength\itemsep{-0.2em}
    \item 1$\ccc$ to 1$\sss$ = 14+ 5542 natural open \\
        open lower suit when same length
    \item 1NT = 10-13 BAL or (4441)
    \item 2$\ccc$ to 2$\sss$ = 10-13 5+ cards natural open \\
        open major suit, then lower suit when same length
        ($\hhh \rightarrow \sss \rightarrow \ccc \rightarrow \ddd$)
    \item 3$\ccc$ onwards = any preemptive scheme you like
\end{itemize}

\subsection{After Strong 1 Open}
\begin{itemize}
    \setlength\itemsep{-0.2em}
    \item \#1 (or XX) = 8+ relay
    \item \#2 to \#6 or double raise = to play
    \begin{itemize}
        \setlength\itemsep{-0.2em}
        \item to 1$\ccc$, SO length is 5555 (1NT $<=$ 4444)
        \item to 1$\ddd$, SO length is 5546 (1NT $<=$ 4435)
        \item to 1$\hhh$, SO length is 5366 (1NT $<=$ 4255)
        \item to 1$\sss$, SO length is 1555
    \end{itemize}
    \item \#7 to \#9 = GI+ splinter
\end{itemize}

\subsubsection{After Relay to Strong 1 Open}
\begin{itemize}
    \setlength\itemsep{-0.2em}
    \item \#1 = 14-16 \\
        after this:
        \begin{itemize}
            \setlength\itemsep{-0.2em}
            \item \#1 = 10+ GF
            \item \#2 to \#5 = to play \\
                apply competitive bidding system
        \end{itemize}
    \item \#2+ = 444766 \\
        after any response:
        \begin{itemize}
            \setlength\itemsep{-0.2em}
            \item \#1 = normal relay
            \item \#2 to \#5 = relay with shortage
        \end{itemize}
\end{itemize}

\subsubsection{After To-Play}
\begin{itemize}
    \setlength\itemsep{-0.2em}
    \item 1NT/2M = 17-19 NF GI \\
        apply competitive bidding system
    \item \#1 (after excluding 1NT/2M) = 20+ \\
        as strong probe if To-Play bid is less specific:
        \begin{itemize}
            \setlength\itemsep{-0.2em}
            \item \#1 = GF relay (resume \nameref{sec:444766})
            \item \#2 to \#6 = to play (double neg)
        \end{itemize}
        as relay ask if To-Play bid is 1NT or specific:
        \begin{itemize}
            \setlength\itemsep{-0.2em}
            \item \#1 = min
            \item \#2+ = GF (as \nameref{sec:444766} or
                \nameref{sec:1nt-to-play} if To-Play bid is 1NT)
        \end{itemize}
    \item \#2+ = 17-19 NF GI \\
        apply competitive bidding system
\end{itemize}

\subsection{After 1NT Open}
This scheme is rather inelastic because upon any overcalls other than double or artificial 2$\ccc$, we should turn to competitive bidding (e.g. Lebensohl).

\begin{itemize}
    \setlength\itemsep{-0.2em}
    \item 2$\ccc$ = GF relay
    \begin{itemize}
        \setlength\itemsep{-0.2em}
        \item 2$\ddd$ = any (4441)
        \item 2$\hhh$ to 3NT = \nameref{sec:2h-3nt-bal}
    \end{itemize}
    \item 2$\ddd$ = any GI
    \item 2$\hhh$ to 3NT = \nameref{sec:2h-3nt-unbal}
\end{itemize}

\subsection{After Intermediate 2 Open}
\begin{itemize}
    \setlength\itemsep{-0.2em}
    \item \#1 = GI+ relay, cont' as \nameref{sec:444766}
    \item \#2 to \#4 = transfer GI+ \\
        short open suit if GF \\
        match transfer first, then $\hhh \rightarrow \sss \rightarrow \ccc \rightarrow \ddd$ \\
        only non-GF response is rebid or follow transfer
        
\end{itemize}

\section{Overcalling Scheme}
\begin{itemize}
    \setlength\itemsep{-0.2em}
    \item X = 12+ takeout or; \\
        18+ when 1-level unavailable
    \item 1$\ccc$ to 1$\sss$ = 14+ 5544 natural open \\
        open lower suit when same length
    \item 1NT = 15-18 BAL
    \item 2$\ccc$ to 2$\sss$ (if 1-level of suit is available) = 10-13 5+ cards natural open
    \item 2$\ccc$ to 2$\sss$ (if 1-level of suit is unavailable) = 13-17 5+ cards natural open (same continuation as 10-13 case)
\end{itemize}

\columnbreak
$$\hhh\sss\ccc\ddd~The~end~\hhh\sss\ccc\ddd$$

\end{multicols}
\end{document}
