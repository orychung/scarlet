\documentclass{article}
\usepackage[utf8]{inputenc}
\usepackage[margin=0.6in]{geometry}
\usepackage[fontsize=12pt]{fontsize}
\usepackage{multicol}
\usepackage{hyperref}
\usepackage{graphicx} % Required for inserting images
\usepackage{CJKutf8} % Required for CJK
\usepackage{bxcjkjatype} % Required for Gothic(gc)
\usepackage[dvipsnames]{xcolor} % Required for \textcolor
\usepackage{pxfonts} % Required for solid diamond / heart symbol
\usepackage{enumitem}
\setlist[enumerate,1]{
    topsep=-0.25em,
    itemsep=-0.25em, left=0em,
    after=\vspace{0.5em}
}
\setlist[itemize]{
    topsep=-1em,
    itemsep=-0.15em, left=0em
}
\setlist[itemize,1]{
    label=$\blacktriangleright$,
    after=\vspace{0.1em}
}
\setlist[itemize,2]{
    label=$\vartriangleright$,
    after=\vspace{0.1em}
}
\setlist[itemize,3]{
    label=$\bullet$
}
\setlist[itemize,4]{
    label=$\ast$
}
\newcommand*{\ccc}{\textcolor{ForestGreen}{\clubsuit}}
\newcommand*{\ddd}{\textcolor{Orange}{\vardiamondsuit}}
\newcommand*{\hhh}{\textcolor{Mahogany}{\varheartsuit}}
\newcommand*{\sss}{\textcolor{Blue}{\spadesuit}}
\newcommand*{\hscd}{\hhh \rightarrow \sss \rightarrow \ccc \rightarrow \ddd}
\newcommand*{\CCC}{\texorpdfstring{$\ccc$}{C}}
\newcommand*{\DDD}{\texorpdfstring{$\ddd$}{D}}
\newcommand*{\HHH}{\texorpdfstring{$\hhh$}{H}}
\newcommand*{\SSS}{\texorpdfstring{$\sss$}{S}}


\title{Just-Enough Two-Suit Opens (JETSO)}
\author{O.Y. Chung $\hhh\sss\ccc\ddd$}
\date{July 2023}

\begin{document}
\maketitle
\begin{multicols}{2}
\begin{CJK*}{UTF8}{gt}

\section{Abstract}
Designed for use in vulnerable side only, Just-Enough Two-Suit Opens (JETSO) brings maximized chance of safer signoffs and extra hint in competitive bidding by showing the hand type and for 2/3-suiter case an extra guarantee to hold one of two specified suits.

\section{Elastic Relay Patterns}
It is a pre-requisite to fully learn RULER before using JETSO because a lot of RULER relay patterns will be reused here.  Following the design of RULER, the main relay is elastic. Minor adjustment to some relay window is needed:
\subsection{NLM3R56}\label{sec:nlm3r56}
By repeating R564 of RULER, the only possible way to have 4 in 2-suiter openings is 4441 because 4 is for balanced hand type. To make best use of 3-suiter hands, a 3-suiter window is added. Together with extra windows for splitting the strength and the suit, there can be 7 windows in a typical relay:
\begin{enumerate}
    \item \textbf{\underline{N}}o 4+ in a \hyperref[sec:determine-n-suit]{determined suit}, then: LM3R56
    \item \textbf{\underline{L}}ow strength, then: 3R56
    \item \textbf{\underline{M}}edium strength, then: R56
    \item \textbf{\underline{3}}-suiter
    \item \textbf{\underline{R}}everser (higher suit is longer)
    \item 5-\textbf{\underline{5}} (inapplicable if bid higher suit first)
    \item \textbf{\underline{6}}-4
\end{enumerate}

\noindent After the 3-suiter step, if there are more than one possible 3rd suit, use a further step 1 for a window for the case of longing the higher suit.

\subsubsection{2/3-Suiter Open Strength Ranges}
\begin{itemize}
    \item 3-suit: low=11-14; high=15+
    \item 2-suit: low=10-13; medium=14-17; high=18+
\end{itemize}

\subsubsection{Determining the "N" Window Suit}\label{sec:determine-n-suit}
4 rules are made for choosing the "N" suit from the 2 potential suits. Take first applicable rule:
\begin{enumerate}
    \item No high chance suit, because:\\
    Bidding N window get less bidding space
    \item No convenient SO suit, because:\\
    Good to bid min at possible SO suit
    \item No matched suit, because:\\
    Good to set asker as declarer
    \item No lower suit ($\ccc<\ddd<\hhh<\sss$), because:\\
    Higher suit has more slam bidding space
\end{enumerate}

\section{Opening Scheme}
\begin{itemize}
    \item 1-suiter (6+ in one suit, 3- in other suits):
    \begin{itemize}
        \item 10-13 $\hhh$: \hyperref[sec:1suit]{2$\hhh$}
        \item 10-13 $\sss$: \hyperref[sec:1suit]{2$\sss$}
        \item 11-13 $\ccc$: 3$\ccc$
        \item 11-13 $\ddd$: 3$\ddd$
        \item 14+: \hyperref[sec:1c]{1$\ccc$}
    \end{itemize}
    \item 2/3-suiter (54+ in 2 suits or 444+ in 3 suits):\\
    ($\ccc$ is never the anchor suit)
    \begin{itemize}
        \item 10+ $\hhh$+$\sss$/$\ccc$: \hyperref[sec:1d]{1$\ddd$}
        \item 10+ $\sss$+$\ddd$/$\ccc$: \hyperref[sec:1h]{1$\hhh$}
        \item 14+ $\ddd$+$\hhh$/$\ccc$: \hyperref[sec:1s]{1$\sss$}
        \item 10-13 $\ccc$+$\ddd$: \hyperref[sec:2cd]{2$\ccc$}
        \item 10-13 $\ddd$+$\hhh$: \hyperref[sec:2cd]{2$\ddd$}
    \end{itemize}
    \item BAL (4333/4432/5332):
    \begin{itemize}
        \item 12-14: \hyperref[sec:1c]{1$\ccc$}
        \item 15-17: \hyperref[sec:1n]{1NT}
        \item 18+: \hyperref[sec:1c]{1$\ccc$}
    \end{itemize}
\end{itemize}

\section{1\CCC\ Open}\label{sec:1c}
\begin{itemize}
    \item 0-10: 1$\ddd$
    \begin{itemize}
        \item 12-14 BAL:
        \begin{itemize}
            \item with 4-5$\sss$s: \hyperref[sec:1c1d1h]{1$\sss$}
            \item with 4-5$\hhh$s: \hyperref[sec:1c1d1h]{1NT}
            \item otherwise: \hyperref[sec:1c1d1h]{1$\hhh$}
        \end{itemize}
        \item 18+ BAL: \hyperref[sec:1c1d1h]{1$\hhh$}
        \item 14-19 1-suiter: 2$\ccc/\ddd/\hhh/\sss$ (same as \nameref{sec:1suit})
        \item 20+ 1-suiter: \hyperref[sec:1c1d1h]{1$\hhh$}
    \end{itemize}
    \item 11+: 1$\hhh$/1$\sss$/1NT/2$\ccc$ 
    for 4+$\hhh/\sss/\ccc/\ddd$
    \begin{itemize}
        \item relay: \#1 (\#1 = min; \#2+ = RULER 444766)
        \item 1-suiter break: \#2-4 for 6+ in another suit ($\hscd$), with shortage, cont' as RULER 111
    \end{itemize}
\end{itemize}

\subsection{After 1\CCC-1\DDD-1\HHH}\label{sec:1c1d1h}
\begin{itemize}
    \item 0-5: \hyperref[sec:1c1d1h1s]{1$\sss$}
    \item 6-10: 1NT/2$\ccc/\ddd/\hhh/\sss$
        (same response scheme as RULER open except all GF)
    \item $<=$ 8 GI with 7+card: 3$\ccc/\ddd/\hhh/\sss$ (all natural)
\end{itemize}

\noindent The same scheme applies for 1$\ccc$-1$\ddd$-1$\sss$/1NT, with these modifications:
\begin{itemize}
    \item the 0-5 and 6-10 ranges are merged
    \item there is a 4-card major raise (for 2$\sss$ over 1$\sss$, it further implies a mild GI)
    \item no more GF responses are present (become some natural / artificial GI)
\end{itemize}

\subsection{After 1\CCC-1\DDD-1\HHH-1\SSS}
\label{sec:1c1d1h1s}
\begin{itemize}
    \item 12-14 or 18-21 BAL: 1NT
    \begin{itemize}
        \item BAL: P
        \item 5+card: 2$\ccc/\ddd/\hhh/\sss$ (18+ may GI over 2M)
    \end{itemize}
    \item 22+ BAL: 2$\ccc$
    \begin{itemize}
        \item \hyperref[sec:1c1d1h1s2c2d]{2$\ddd$} = waiting
        \item 2$\hhh$ to 3NT = RULER 2$\hhh$ to 3NT Unbalanced
    \end{itemize}
    \item 20+ 1-suiter: 2$\ddd/\hhh/\sss$/NT for $\hhh/\sss/\ccc/\ddd$ (same as \nameref{sec:1suit})
\end{itemize}

\subsection{After 1\CCC-1\DDD-1\HHH-1\SSS-2\CCC-2\DDD}
\label{sec:1c1d1h1s2c2d}
\begin{itemize}
    \item 22-24: 2$\hhh$ (or 2$\sss$ if $\sss$ is 2-card longer than $\hhh$) \\
        same as NT overcall, any extra 2M is NF
    \item 25-26: 2NT (same as NT overcall)
    \item 27+: 3$\ccc$-3NT (RULER 1NT GI)
\end{itemize}

\subsection{Relay Breaks in Cases Like \texorpdfstring{\\}{}
1\CCC-1\HHH-1\SSS-1NT}
In 1$\ccc$-1$\hhh$-1$\sss$-1NT, the 1NT bid shows a minimum hand of 11-13. After this 1NT, \#1 is GF relay while \#2 to \#6 are natural SO. Pass cannot be used because the maximum of 1NT is still high enough for a game. After an SO bid is chosen, partner can bid further using a competitive bidding scheme (G3LBS).

\subsection{Strength Adjustment for Positive Responses}
\begin{tabular}{c|c|c}
    \hline
    Case & Min & Max \\ \hline
    Not passed & 11-13 = F1 & 14+ \\ \hline
    Passed & 9-10 & 11 = 13+ to relay \\ \hline
    Overcalled & 9-11 & 12+ \\ \hline
\end{tabular}

\noindent The table above summarizes the different strength range to use in different cases. It is important to note that:
\begin{itemize}
    \item Min case is forcing only when not passed. For other cases, no G3LBS is applicable because any rebid will just be corrections.
    \item For passed case, relaying the first positive bid requires extra strength (13+). No more 1-suiter relay break can be used.
\end{itemize}

\section{2-Suiter Opens}
General approach is to use \#1 to relay, and remaining non-jump bids to escape.

\subsection{1\DDD\ Open}\label{sec:1d}
\begin{itemize}
    \item 1$\hhh$ = 9+, relay, cont' as \nameref{sec:nlm3r56}
    \item \hyperref[sec:1d-escape]{1$\sss$} = 0-8 fit $\sss\ccc$ / fit $\hhh\ccc$ / long $\sss$
    \item \hyperref[sec:1d-escape]{1NT} = 0-8 $\sss>\hhh>\ccc$
    \item \hyperref[sec:1d-escape]{2$\ccc$} = 0-8 request to stop at 5+card suit
    \item 2$\ddd$ = 0-8 6+$\ddd$s, G3LBS follows
    \item 2$\hhh$ = 0-8 3+$\hhh$s, G3LBS follows
\end{itemize}

\subsubsection{After 1\DDD-1\HHH-1\SSS}
\begin{itemize}
    \item 1NT(\#1) = 11+, cont' as \hyperref[sec:nlm3r56]{LM3R56}
    \item \#2 to \#6 = 9-10 SO, pass or cont' as \hyperref[sec:nlm3r56]{3R56}
\end{itemize}

\subsubsection{After 1\DDD-1\HHH-1NT}
\begin{itemize}
    \item 2$\ccc$(\#1) = 14+, cont' as \hyperref[sec:nlm3r56]{3R56}
    \item 2$\ddd$ = 12-13 GI, G3LBS follows
    \item others = SO, all natural
\end{itemize}

\subsubsection{After 1\DDD-1\HHH-2\CCC}
\begin{itemize}
    \item 2$\ddd$(\#1) = 10+, cont' as \hyperref[sec:nlm3r56]{R56}
    \item others = SO, G3LBS follows
\end{itemize}

\subsubsection{After 1\DDD-1\SSS/1NT/2\CCC}\label{sec:1d-escape}
When partner escapes to lower than 2$\ddd$, both 2$\ddd$ and 2NT can be used to show unbounded strong hand, providing extra chance to resume relay.
\begin{itemize}
    \item 2$\ddd$ = 17+ with unlikely fit suit
    \item 2NT = 17+ with likely fit suit
    \item 3$\ccc/\hhh/\sss$ = 15-16 6+card NF
    \item others = correction
\end{itemize}

\noindent To the 17+ strong bids, \#1 is GF relay (6-8), other bids are SO. The only exception is when \#1 = 3$\ccc$ is a shown suit, it will become SO and 3$\ddd$ is used for relay. On SO here, if still below 2NT, G3LBS will be used.

\noindent Example of using 3$\ddd$ for relay occurs at 1$\ddd$-2$\ccc$-2NT(likely fit suit = $\ccc$):
\begin{itemize}
    \item 3$\ccc$ = SO
    \item 3$\ddd$ = 6-8 GF relay
\end{itemize}

\noindent Example of using G3LBS occurs at 1$\ddd$-1$\sss$-2$\ddd$:
\begin{itemize}
    \item 2$\hhh$ = 6-8 GF relay
    \item 2$\sss$ = SO, G3LBS follows
\end{itemize}

\subsection{1\HHH\ Open}\label{sec:1h}
Most responses are symmetric to 1$\ddd$ open, but a difference is made to the changed escape scheme.
\begin{itemize}
    \item 1$\sss$ = 9+, relay, cont' as \nameref{sec:nlm3r56}
    \item \hyperref[sec:1h-escape]{1NT} = 0-8 fit $\sss\ccc$
    \item \hyperref[sec:1h-escape]{2$\ccc$} = 0-8 request to stop at 5+card suit
    \item \hyperref[sec:1h-escape]{2$\ddd$} = 0-8 $\ddd>\sss>\ccc$
    \item 2$\hhh$ = 0-8 6+$\hhh$s, G3LBS follows
    \item 2$\sss$ = 0-8 3+$\sss$s, G3LBS follows
\end{itemize}

\subsubsection{After 1\HHH-1NT/2\CCC/2\DDD}\label{sec:1h-escape}
\begin{itemize}
    \item 2$\hhh$ = 17+ with unlikely fit suit
    \item 2NT = 17+ with likely fit suit
    \item 3$\ccc/\ddd/\sss$ = 15-16 6+card NF
    \item others = correction
\end{itemize}

\subsection{1\SSS\ Open}\label{sec:1s}
Escape space is further reduced. Relay strength requirement is lowered by 1 point because the open promises 14+.
\begin{itemize}
    \item 1NT = 8+, relay, cont' as \hyperref[sec:nlm3r56]{NM3R56}
    \item \hyperref[sec:1s-escape]{2$\ccc$} = 0-7 request to stop at 5+card suit
    \item \hyperref[sec:1s-escape]{2$\ddd$} = 0-7 avoiding a $\ccc$ stop
    \item \hyperref[sec:1s-escape]{2$\hhh$} = 0-7 5+$\hhh$s
    \item 2$\sss$ = 0-7 5+$\sss$s, G3LBS follows
\end{itemize}

\subsubsection{After 1\SSS-2\CCC/2\DDD/2\HHH}\label{sec:1s-escape}
\begin{itemize}
    \item 2$\sss$ = 18+ with unlikely fit suit
    \item 2NT = 18+ with likely fit suit
    \item 3$\ccc/\ddd/\hhh$ = 16-17 6+card NF
    \item others = correction
\end{itemize}

\subsection{2\CCC/\DDD\ Open}\label{sec:2cd}
\begin{itemize}
    \item \#1 = correction
    \item \#2 = 14+ GF relay, cont' as \hyperref[sec:nlm3r56]{R56}
    \item 2NT = any GI, \#1 = min
    \item others = SO
\end{itemize}

\section{Other Opens}
\subsection{1NT Open}\label{sec:1n}
RULER 1NT open scheme with the advancement to allow $\hhh$ SO. Such advancement can be used because there is no 4441 cases and it is valuable to have better SO option when vulnerable.

\subsection{1-Suiter Open at 2-Level}\label{sec:1suit}
\begin{itemize}
    \item \#1 = GI+ relay, cont' as RULER 111
    \item \#2-\#4 = transfer GI+ (similar to RULER)
\end{itemize}

$$\hhh\sss\ccc\ddd~The~end~\hhh\sss\ccc\ddd$$

\end{CJK*}
\end{multicols}
\end{document}
