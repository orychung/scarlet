\documentclass{article}
\usepackage[utf8]{inputenc}
\usepackage[margin=0.6in]{geometry}
\usepackage[fontsize=12pt]{fontsize}
\usepackage{multicol}
\usepackage{hyperref}
\usepackage{graphicx} % Required for inserting images
\usepackage{CJKutf8}
\usepackage[dvipsnames]{xcolor} % Required for \textcolor
\usepackage{pxfonts} % Required for solid diamond / heart symbol
\newcommand*{\ccc}{\textcolor{ForestGreen}{\clubsuit}}
\newcommand*{\ddd}{\textcolor{Orange}{\vardiamondsuit}}
\newcommand*{\hhh}{\textcolor{Mahogany}{\varheartsuit}}
\newcommand*{\sss}{\textcolor{Blue}{\spadesuit}}

\title{Generalized 3-way Lebensohl (G3LBS)}
\author{O.Y. Chung $\hhh\sss\ccc\ddd$}
\date{July 2023}

\begin{document}

\begin{CJK*}{UTF8}{gbsn}
\end{CJK*}
\maketitle

\begin{multicols}{2}

\section{Abstract}
Existing variants of Lebensohl focused on 3 cases: 1NT overcalled; double over 2-level open; and non-GF reverse. Some also forgoes invitational cases when suit is lower than opponents' suit. The generalization here extends it to apply whenever the bidding is below 2NT, no better system on (e.g. relays), and the bidding is possible to land on partial and game contracts, resulting in a memory-saving and effective competitive auction approach.

\section{The 3-way Plan}
Signoff (SO), invitational (GI), and game-forcing (GF) suit bids are generally avaiable in this scheme.

\begin{enumerate}
    \setlength\itemsep{-0.2em}
    \item SO: bid suit at 2-level, or one step below the 3-level suit bid.
    \item GI: bid 2NT, then stop at the suit at 3-level \\
    when it is a non-compulsory bid, GI with $\ccc$ case is included
    \item GF: bid suit at 3-level, or perform the symbollic GF sequences of Lebenshol:
    \begin{itemize}
        \setlength\itemsep{-0.2em}
        \item no 4M nor stopper: 3NT
        \item 4M but no stopper: cuebid
        \item no 4M but stopper: 2NT, then 3NT
        \item 4M and stopper: 2NT, then cuebid
    \end{itemize}
    This can be regarded as a rule to bid 2NT when there is stopper in opponents' suit, with exception when opponents has no suit or $\ccc$ suit.
\end{enumerate}

\subsection{Typical Schemes}
\subsubsection{2$\sss$-X-P-?}
\begin{tabular}{c|c|c|c}
    \hline
    Suit & SO & GI & FG \\ \hline\hline
    $\ccc$ & 2NT, P & (n/a) & 3$\ccc$ \\ \hline
    $\ddd$ & 3$\ccc$, P & 2NT, 3$\ddd$ & 3$\ddd$ \\ \hline
    $\hhh$ & 3$\ddd$, P & 2NT, 3$\hhh$ & 3$\hhh$ \\ \hline
    4$\hhh$ & 3$\ddd$, P & 2NT, 3$\hhh$ & (2NT,) 3$\sss$ \\ \hline
    no M & (SO at minor) & (n/a) & (2NT,) 3NT \\ \hline
\end{tabular}

\subsubsection{2$\hhh$-X-P-?}
\begin{tabular}{c|c|c|c}
    \hline
    Suit & SO & GI & FG \\ \hline\hline
    $\ccc$ & 2NT, P & (n/a) & 3$\ccc$ \\ \hline
    $\ddd$ & 3$\ccc$, P & 2NT, 3$\ddd$ & 3$\ddd$ \\ \hline
    $\sss$ & 2$\sss$ & 2NT, 3$\sss$ & 3$\sss$ \\ \hline
    4$\sss$ & 2$\sss$ & 2NT, 3$\sss$ & (2NT,) 3$\hhh$ \\ \hline
    no M & (SO at minor) & (n/a) & (2NT,) 3NT \\ \hline
\end{tabular}

\subsubsection{2$\ddd$-X-P-?}
\begin{tabular}{c|c|c|c}
    \hline
    Suit & SO & GI & FG \\ \hline\hline
    $\ccc$ & 2NT, P & (n/a) & 3$\ccc$ \\ \hline
    $\hhh$ & 2$\hhh$ & 2NT, 3$\hhh$ & 3$\hhh$ \\ \hline
    $\sss$ & 2$\sss$ & 2NT, 3$\sss$ & 3$\sss$ \\ \hline
    4$\hhh$ & 2$\hhh$ & 2NT, 3$\hhh$ & (2NT,) 3$\ddd$ \\ \hline
    4$\sss$ & 2$\sss$ & 2NT, 3$\sss$ & (2NT,) 3$\ddd$ \\ \hline
    no M & (SO at minor) & (n/a) & (2NT,) 3NT \\ \hline
\end{tabular}

\subsubsection{2$\ccc$-X-P-?}
\begin{tabular}{c|c|c|c}
    \hline
    Suit & SO & GI & FG \\ \hline\hline
    $\ddd$ & 2$\ddd$ & 2NT, 3$\ddd$ & 3$\ddd$ \\ \hline
    $\hhh$ & 2$\hhh$ & 2NT, 3$\hhh$ & 3$\hhh$ \\ \hline
    $\sss$ & 2$\sss$ & 2NT, 3$\sss$ & 3$\sss$ \\ \hline
    4$\hhh$ & 2$\hhh$ & 2NT, 3$\hhh$ & 3$\ccc$ \\ \hline
    4$\sss$ & 2$\sss$ & 2NT, 3$\sss$ & 3$\ccc$ \\ \hline
    no M & (SO at minor) & (n/a) & (2NT,) 3NT \\ \hline
\end{tabular}

\section{Generalizations}
\subsection{Super Accepts}
For SO or $\ccc$ GI, there is a need to super accept to ensure bidding does not quit by partner's planned pass. All super accept bids are GF.
\begin{itemize}
    \setlength\itemsep{-0.2em}
    \item \#2: waiting (implies major fit if lack room to check the possible major, otherwise bit 3NT)
    \item cuebid: stopper-ask
    \item 3NT: no major fit
    \item more: natural
\end{itemize}

\noindent Take 1NT-2$\sss$-? as example: \\
\begin{tabular}{c|c|c|c|c}
    \hline
    Bid & 3$\ddd$ & 3$\hhh$ & 3$\sss$ & 3NT \\ \hline\hline
    2NT & waiting & $\hhh$ & s/a & (avoided) \\ \hline
    3$\ccc$ & (normal) & waiting & s/a & (avoided) \\ \hline
    3$\ddd$ & (n/a) & (normal) & $\hhh$ fit & no $\hhh$ fit \\ \hline
\end{tabular}

\subsection{Cuebid on $\ccc$ Suit}
When $\ccc$ is the suit to cuebid, to replenish the 2NT slow pattern, the 3$\ddd$ response can be used:
\begin{itemize}
    \setlength\itemsep{-0.2em}
    \item 3$\ddd$: $\ccc$ stopper
    \item 3$\hhh$: no $\ccc$ stopper, check $\hhh$ fit
    \item 3$\sss$: no $\ccc$ stopper, check $\sss$ fit
    \item 3NT: no $\ccc$ stopper, no M fit
\end{itemize}

\subsection{No Suit to Cuebid}
When opponents do not show any 5+card suit, treat 3$\ccc$ as opponents' suit. A $\ccc$ suit strong case can be merged into 3$\ccc$, using 3$\ddd$ response to show preference of 5$\ccc$ over 3NT.
\begin{itemize}
    \setlength\itemsep{-0.2em}
    \item 3$\ddd$: 5$\ccc$ preference
    \item 3$\hhh$: no 5$\ccc$ preference, check $\hhh$ fit
    \item 3$\sss$: no 5$\ccc$ preference, check $\sss$ fit
    \item 3NT: no 5$\ccc$ preference, no M fit
\end{itemize}

\noindent The no major GF case can also include an extra breakdown of meaning:
\begin{itemize}
    \setlength\itemsep{-0.2em}
    \item 2NT-then-3NT if honors distributes evenly across suits
    \item direct 3NT if honors distributes unevenly across suits
\end{itemize}

\subsection{Suit Selection}
When another suit is shown, and the transferred suit is not responder's suit, responder may choose a previously shown suit, without a GF meaning. This signoff has higher priority than any super accepts.

\noindent Take 1$\sss$-P-2$\ddd$-2$\hhh$-? as example: \\
\begin{tabular}{c|c|c|c|c}
    \hline
    Bid & 3$\ddd$ & 3$\hhh$ & 3$\sss$ & 3NT \\ \hline\hline
    2NT & $\ddd$ SO & $\sss$ fit & $\sss$ SO & no $\sss$ fit \\ \hline
    3$\ccc$ & $\ddd$ NF & $\sss$ fit & $\sss$ SO & no $\sss$ fit \\ \hline
\end{tabular}

\noindent Take 1$\hhh$-P-2$\ddd$-2$\sss$-? as example: \\
\begin{tabular}{c|c|c|c|c}
    \hline
    Bid & 3$\ddd$ & 3$\hhh$ & 3$\sss$ & 3NT \\ \hline\hline
    2NT & $\ddd$ SO & $\hhh$ SO & $\hhh$ fit & no $\hhh$ fit \\ \hline
    3$\ccc$ & $\ddd$ NF & $\hhh$ SO & $\hhh$ fit & no $\hhh$ fit \\ \hline
    3$\ddd$ & (n/a) & $\hhh$ NF & $\hhh$ fit & no $\hhh$ fit \\ \hline
\end{tabular}

\noindent Take 1$\ccc$-P-1$\ddd$-2$\sss$-? as example: \\
\begin{tabular}{c|c|c|c|c}
    \hline
    Bid & 3$\ddd$ & 3$\hhh$ & 3$\sss$ & 3NT \\ \hline\hline
    2NT & $\ddd$ SO & $\hhh$ GF & s/a & catch all \\ \hline
    3$\ccc$ & $\ddd$ NF & $\hhh$ GF & s/a & catch all \\ \hline
    3$\ddd$ & (n/a) & $\hhh$ NF & $\hhh$ fit & no $\hhh$ fit \\ \hline
\end{tabular}

\subsection{Enhance Major Fit Test from 4-card to Sub-optimal Length}
Optimal length for a major suit is extended to be:
\begin{itemize}
    \setlength\itemsep{-0.2em}
    \item new suit: 5+ cards
    \item partner min length: 8 - (min length)
    \item partner max length: 8 - (max length) + 1
    \item self max length: max length
    \item self min length: min length + 2
\end{itemize}
Sub-optimal length is optimal length minus 1. Examples:
\begin{itemize}
    \setlength\itemsep{-0.2em}
    \item 1$\hhh$-P-1$\sss$-2$\ccc$-?, to show GF:
    \begin{itemize}
        \setlength\itemsep{-0.2em}
        \item 7+$\hhh$s: 3$\hhh$
        \item 4+$\sss$s: 3$\sss$
        \item 6$\hhh$s or 3$\sss$s: 3$\ccc$
    \end{itemize}
    \item 1$\hhh$-P-1$\sss$-2$\ccc$-P-P-?, to show GF:
    \begin{itemize}
        \setlength\itemsep{-0.2em}
        \item 3+$\hhh$s: 3$\hhh$
        \item 6+$\sss$s: 3$\sss$
        \item 2$\hhh$s or 5$\sss$s: 3$\ccc$
    \end{itemize}
    \item 1NT-P-2$\ccc$-P-2$\ddd$-2$\sss$-?, to show GF:
    \begin{itemize}
        \setlength\itemsep{-0.2em}
        \item 6+$\hhh$s: 3$\hhh$
        \item 5$\hhh$s: 3$\sss$
    \end{itemize}
\end{itemize}

\section{Situations Explained}
\subsection{Opening Overcalled}
The most basic and common examples are direct overcalls:

\subsection{Relay Break}
\subsection{Michaels Cuebid}
\subsection{Opening overcalled}

\columnbreak
$$ -end- $$

\end{multicols}
\end{document}
